\documentclass{article}
\usepackage[utf8]{inputenc}

\title{Introduction to Quantum Computing}
\author{Jane Researcher}

\begin{document}

\maketitle

\section{Introduction}

Quantum computing represents a fundamental shift in how we process information. Unlike classical computers that use \textbf{bits} (0 or 1), quantum computers use \textit{qubits} which can exist in superposition.

This paper explores the basic principles of quantum computing and its potential applications.

\section{Quantum Principles}

\subsection{Superposition}

A qubit can exist in multiple states simultaneously. This property, called \emph{superposition}, is mathematically represented as a linear combination of basis states.

Key properties include:

\begin{itemize}
\item Qubits can be in state 0, state 1, or both simultaneously
\item Measurement collapses the superposition
\item Enables parallel computation
\end{itemize}

\subsection{Entanglement}

When qubits become entangled, the state of one qubit instantly affects the state of another, regardless of distance.

\section{Applications}

\subsection{Cryptography}

Quantum computers could break many current encryption schemes, but also enable quantum-safe cryptography.

\subsection{Drug Discovery}

The following areas show promise:

\begin{enumerate}
\item Molecular simulation
\item Protein folding
\item Chemical reaction modeling
\end{enumerate}

\section{Conclusion}

Quantum computing is still in its early stages, but shows tremendous potential. The field requires advances in both hardware and algorithms to reach practical utility.

\end{document}
