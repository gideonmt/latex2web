\documentclass{article}
\usepackage[utf8]{inputenc}
\usepackage{amsmath}
\usepackage{graphicx}
\usepackage{listings}

\title{Introduction to Quantum Computing}
\author{Jane Researcher}

\begin{document}

\maketitle

\section{Introduction}

Quantum computing represents a fundamental shift in how we process information. Unlike classical computers that use \textbf{bits} (0 or 1), quantum computers use \textit{qubits} which can exist in superposition.

This paper explores the basic principles of quantum computing and its potential applications.

\section{Quantum Principles}

\subsection{Superposition}

A qubit can exist in multiple states simultaneously. This property, called \emph{superposition}, is mathematically represented as:

\[
|\psi\rangle = \alpha|0\rangle + \beta|1\rangle
\]

where $\alpha$ and $\beta$ are complex probability amplitudes satisfying $|\alpha|^2 + |\beta|^2 = 1$.

Key properties include:

\begin{itemize}
\item Qubits can be in state 0, state 1, or both simultaneously
\item Measurement collapses the superposition
\item Enables parallel computation
\end{itemize}

\subsection{Entanglement}

When qubits become entangled, the state of one qubit instantly affects the state of another, regardless of distance. A maximally entangled Bell state is:

\[
|\Phi^+\rangle = \frac{1}{\sqrt{2}}(|00\rangle + |11\rangle)
\]

\section{Quantum Gates}

Common quantum gates and their matrix representations:

\begin{table}[h]
\centering
\begin{tabular}{|l|c|c|}
\hline
Gate & Symbol & Matrix \\
\hline
Pauli-X & X & $\begin{pmatrix} 0 & 1 \\ 1 & 0 \end{pmatrix}$ \\
Pauli-Y & Y & $\begin{pmatrix} 0 & -i \\ i & 0 \end{pmatrix}$ \\
Pauli-Z & Z & $\begin{pmatrix} 1 & 0 \\ 0 & -1 \end{pmatrix}$ \\
Hadamard & H & $\frac{1}{\sqrt{2}}\begin{pmatrix} 1 & 1 \\ 1 & -1 \end{pmatrix}$ \\
\hline
\end{tabular}
\end{table}

\section{Example: Quantum Teleportation}

Here's a simple quantum circuit in pseudocode:

\begin{verbatim}
def quantum_teleportation(psi):
    # Create Bell pair
    bell_pair = create_bell_state()
    
    # Alice's operations
    cx(psi, bell_pair[0])
    h(psi)
    
    # Measure
    m1 = measure(psi)
    m2 = measure(bell_pair[0])
    
    # Bob's correction
    if m2: x(bell_pair[1])
    if m1: z(bell_pair[1])
    
    return bell_pair[1]
\end{verbatim}

\section{Applications}

\subsection{Cryptography}

Quantum computers could break many current encryption schemes, but also enable quantum-safe cryptography.

\subsection{Drug Discovery}

The following areas show promise:

\begin{enumerate}
\item Molecular simulation
\item Protein folding
\item Chemical reaction modeling
\end{enumerate}

\section{Conclusion}

Quantum computing is still in its early stages, but shows tremendous potential. The field requires advances in both hardware and algorithms to reach practical utility.

\end{document}
